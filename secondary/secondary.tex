\documentclass{article}

\usepackage[paperheight=3in,paperwidth=5in,margin=0.5cm]{geometry}

\usepackage{amsmath}
\usepackage[arrowdel]{physics}
\usepackage{siunitx}

\AtBeginDocument{\RenewCommandCopy\qty\SI}

\begin{document}
   

1) Derive maximum height of projectile launched at angle $\theta$ and initial speed $v_{0}$.
\begin{itemize}    
    \item Use equation for $\Delta v^2$ in terms of $g$ and $y$.
\end{itemize}

\newpage

2) Derive trajectory of projectile (i.e. $y(x)$)

\begin{itemize}
    \item solve for $t$ from $x$ equation
    \item plug $t$ into $y$ equation and simplify
\end{itemize}

\newpage 

3) Derive range of projectile launched at angle $\theta$ and initial speed $v_{0}$ (assume the final height is the same as the initial height).
\begin{itemize}
\item Use projectile motion trajectory equation to solve for change in $x$
\end{itemize}

\newpage


4) Derive a formula relating $\Delta v^2$ to an object's acceleration $a$ and change in position $\Delta x$ (1D).
\begin{itemize}
    \item Use work-energy theorem to derive this
\end{itemize}

\newpage

5) Work out the units of the gravitational constant $G$ (MKS).
\begin{itemize}
    \item Use Newton's law of gravity to deduce the units 
\end{itemize}

\newpage

6) Write down Maxwell's equations for electrostatics and magnetostatics (separately).
\begin{itemize}
    \item Consider changes of electric and magnetic fluxes or charge/current distributions behaviors with time. 
\end{itemize}

\newpage

7) Derive Poisson's equation and Laplace's equation from the electrostatic Maxwell equations.
\begin{itemize}
    \item Substitute the definition of the electrostatic potential into Maxwell's equations for Poisson's equation
    \item Assume that $\rho = 0$ for Laplace's equation
\end{itemize}

\newpage

8) Derive the electric field from an infinite plane with surface charge density $\sigma$.
\begin{itemize}
    \item Use Gauss's law with a thin pillbox with one face outside and another inside the plane (faces are of area $A$) 
\end{itemize}

\newpage

9) Derive the electric field from a line charge with linear density $\lambda$.
\begin{itemize}
    \item Use Gauss's law with a cylinder of radius $s$ and length $\ell$ (the length cancels)
\end{itemize}

\newpage

10) Derive the force for a magnetic field on a small piece of current carrying wire.
\begin{itemize}
    \item Consider the small charge element from the current 
\end{itemize}

\newpage

11) Derive the vector potential definition as a consequence of the Maxwell Equation $\grad \vdot \va*{B} = 0$

\newpage

12) Derive the magnetic field for an (1) infinite straight wire, (2) a solenoid, and (3) a toroid
\begin{itemize}
    \item Use the appropriate paths to integrate using $\displaystyle \int \va*{B} \vdot \dd{\va*{l}} = \mu_0 I_{\rm enc}$ 
    \item For the toroid let the infinite wire coil onto itself $N$ times 
\end{itemize}

\newpage

13) Derive the radius and frequency of cyclotron motion (circular motion of charged particle in magnetic field)
\begin{itemize}
    \item Use the Lorentz force law for a charge in a magnetic field and centripetal acceleration formula for the radius
    \item then use velocity-angular frequency relation to find the angular frequency
\end{itemize}

\newpage

14) Derive the characteristic times for (1) $RL$ and (2) $RC$ circuits and (3) the resonant frequency of an $LC$ circuit.



\end{document}



