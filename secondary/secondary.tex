\documentclass{article}

\usepackage[paperheight=3in,paperwidth=5in,margin=0.5cm]{geometry}

\usepackage{amsmath}
\usepackage[arrowdel]{physics}
\usepackage{siunitx}

\AtBeginDocument{\RenewCommandCopy\qty\SI}

\begin{document}
   

1) Derive maximum height of projectile launched at angle $\theta$ and initial speed $v_{0}$.
\begin{itemize}    
    \item Use equation for $\Delta v^2$ in terms of $g$ and $y$.
\end{itemize}

\newpage

2) Derive trajectory of projectile (i.e. $y(x)$)

\begin{itemize}
    \item solve for $t$ from $x$ equation
    \item plug $t$ into $y$ equation and simplify
\end{itemize}

\newpage 

3) Derive range of projectile launched at angle $\theta$ and initial speed $v_{0}$ (assume the final height is the same as the initial height).
\begin{itemize}
\item Use projectile motion trajectory equation to solve for change in $x$
\end{itemize}

\newpage


4) Derive a formula relating $\Delta v^2$ to an object's acceleration $a$ and change in position $\Delta x$ (1D).
\begin{itemize}
    \item Use work-energy theorem to derive this
\end{itemize}

\newpage

5) Work out the units of the gravitational constant $G$ (MKS).
\begin{itemize}
    \item Use Newton's law of gravity to deduce the units 
\end{itemize}



\end{document}



