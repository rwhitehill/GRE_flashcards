\documentclass{article}

\usepackage[paperheight=3in,paperwidth=5in,margin=0.5cm]{geometry}

\usepackage{amsmath}
\usepackage[arrowdel]{physics}
\usepackage{siunitx}

\AtBeginDocument{\RenewCommandCopy\qty\SI}

\begin{document}
   
1)

\begin{align*}
    h = \frac{v_0^2 \sin^2{\theta}}{2g} 
\end{align*}

\newpage

2)

\begin{align*}
    \Delta y = \tan{\theta} \Delta x - \frac{g \Delta x^2}{2 v_0^2 \cos^2{\theta}}
\end{align*}

\newpage

3) 

\begin{align*}
   R = \frac{v_0^2 \sin(2\theta)}{g} 
\end{align*}

\newpage

4) 
\begin{align*}
   \Delta v^2 = 2 a \Delta x 
\end{align*}

\newpage

5) 

\begin{align*}
    [G] = \SI[per-mode=fraction]{}{\m\cubed\per\kilo\gram\per\s\squared}
\end{align*}

\newpage

6)

Electrostatics:
\begin{align*}
    \grad \vdot \va*{E} = \frac{\rho}{\epsilon_0} &\Leftrightarrow \int \va*{E} \vdot \dd{\va*{a}} = \frac{Q_{\rm enc}}{\epsilon_0} \\
    \grad \cross \va*{E} = 0 &\Leftrightarrow \int \va*{E} \vdot \dd{\va*{l}} = 0
\end{align*}

Magnetostatics:
\begin{align*}
    \grad \vdot \va*{B} = 0 &\Leftrightarrow \int \va*{B} \vdot \dd{\va*{a}} = 0 \\
    \grad \cross \va*{B} = \mu_0 \va*{J} &\Leftrightarrow \int \va*{B} \vdot \dd{\va*{l}} = \mu_0 I_{\rm enc}
\end{align*}

\newpage

7) 

\begin{align*}
    &{\rm Poisson}:~ \laplacian V = \frac{\rho}{\epsilon_0} \\
    &{\rm Laplace}:~ \laplacian V = 0
\end{align*}

\newpage

8) 

\begin{align*}
   \va*{E} = \frac{\sigma}{2 \epsilon_0} \vu*{n} 
\end{align*}

\newpage

9) 

\begin{align*}
   \va*{E} = \frac{\lambda}{2 \pi \epsilon_0 s} \vu*{s}
\end{align*}

\newpage

10)

\begin{align*}
    \dd{\va*{F}} = I \dd{\va*{l}} \cross \va*{B}    
\end{align*}

\newpage

11)

\begin{align*}
   \va*{B} = \grad \cross \va*{A} 
\end{align*}

\newpage

12) 

\begin{align*}
    &{\rm straight~wire}:~ B = \frac{\mu_0 I}{2 \pi s} \\
    &{\rm solenoid}:~ B = \mu_0 I n \\
    &{\rm toroid}:~ B = \frac{\mu_0 I N}{2 \pi s} 
\end{align*}

\newpage

13)

\begin{align*}
    R &= \frac{mv}{qB}  \\
    \omega &= \frac{qB}{m}
\end{align*}

\newpage

14)

\begin{enumerate}
    \item $\displaystyle \tau_{\rm RL} = \frac{L}{R}$ 
    \item $\displaystyle \tau_{\rm RC} = \frac{1}{RC}$
    \item $\displaystyle \omega_{\rm res} = \frac{1}{\sqrt{LC}}$
\end{enumerate}


\end{document}


